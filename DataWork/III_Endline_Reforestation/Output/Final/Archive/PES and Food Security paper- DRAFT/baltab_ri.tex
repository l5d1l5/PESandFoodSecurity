\begin{table}[h!]
\caption{Mean Comparison of Food Security Indicators across Treatment Groups}
\centering
\begin{adjustbox}{max width=\textwidth}
\begin{tabular}{lcccccc} \hline \hline
                  &                      & (1)         &                               & (2)           &         & Randomization \\            
                  &                      & Control &                           & Treatment & t-test  & inference         \\            
 Variable & N/[Clusters] & Mean/SE &  N/[Clusters] & Mean/SE   & p-value & p-value       \\ \hline 
                             Total (nominal) food consumption expenditures in FCFA (past 7 days) & 271 & 11097.077 & 303 & 12940.488 & 0.041 & 0.042 \\    &  & (655.157) &  & (690.208) &  &  \\  (LOG) Total (nominal) food consumption expenditures in FCFA (past 7 days) & 271 & 9.022 & 303 & 9.147 & 0.033 & 0.026 \\   &  & (0.046) &  & (0.047) &  &  \\  Household Food Insecurity Access Scale (HFIAS) Score & 271 & 1.517 & 303 & 0.931 & 0.000 & 0.000 \\   &  & (0.132) &  & (0.108) &  &  \\  HFIA\_cat==Food secure (0/1) & 271 & 0.594 & 303 & 0.736 & 0.000 & 0.000 \\   &  & (0.030) &  & (0.025) &  &  \\  HFIA\_cat==Severely food insecure (0/1) & 271 & 0.151 & 303 & 0.063 & 0.000 & 0.000 \\   &  & (0.022) &  & (0.014) &  &  \\  Household Dietary Diversity Score (HDDS) & 271 & 3.845 & 303 & 3.878 & 0.781 & 0.799 \\   &  & (0.085) &  & (0.082) &  &  \\       \hline
\multicolumn{7}{@{}p{1.2\textwidth}}
{\textit{Notes}: We show both standard p-values and p-values computed by randomization inference (RI) with 1000 repetitions.}
\end{tabular}
\end{adjustbox}
\end{table}